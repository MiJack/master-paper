\newpage
\vspace{-1cm}
\chapter*{\zihao{-2}\heiti{ABSTRACT}}
\vspace{-0.5cm}
%With the close coordination of calculation, communication and control technology, Cyber-Physical Energy System(CPES) realizes the organic integration of network infrastructure and physical infrastructure, which uses intelligent control technology to effectively manage system energy.
%CPES exists in complex physical environment and interacts with various random behaviors in the environment, in which continuous behaviors and discrete behaviors coexist. Therefore, it is stochastic and hybrid.
Cyber-Physical Energy System(CPES) is a kind of complex system with stochastic and hybrid features.
Smart building is a typical CPES, which changes the indoor environment parameters through the intelligent control of functional components, to provide a comfortable living environment for human beings. Also, it strives to reduce overall energy consumption of the system.

In the process of traditional development methods, systems are tested after designed and developed, and it is often too costly and time-consuming to modify systems. Thus, it is of great importance to use Model-Driven Development(MDD) method in order to find errors and inconsistencies in systems in the early stage. However, the modeling of complex CPESs such as smart buildings faces many challenges: 1) In view of the characteristics of CPES, combined with the features of smart buildings, how to effectively reuse the knowledge in this field, and guide the construction of design models? 2)How to modify the standard modeling language MARTE/UML to build system design models? 3)How to construct the executable models of CPES, and to analyze energy consumption and other properties?
To address the above problems, we propose a modeling and analysis method for energy consumption of smart buildings. The main contributions of this paper are as follows:

First of all, ontology of smart buildings is given based on the building information model and the main steps of ontology construction methods. The domain ontology includes the basic concepts of smart buildings, the relationships among concepts and the constraints. Besides, we also point out how to use it to guide system modeling.

Secondly, in view of CPES's energy-aware, stochastic and hybrid features, we extend the MARTE/UML modeling language: 1)Two typical stochastic behaviors are defined; 2)Data types and expressions in MARTE are extended; 3)The meta model of extended class diagrams, sequence diagrams, and state charts are presented to support modeling of energy consumption, stochastic, and hybrid behaviors.

Although MARTE/UML can realize the multi-view modeling of systems in an intuitive way, it is almost impossible to achieve model validation and evaluation because of its semi formal attributes. To solve this problem, we extend the Stochastic Hybrid Automata to Energy Stochastic Hybrid Automata(ESHA) based on energy consumption, and define its syntax and semantics. Besides, mapping rules from MARTE/UML state charts to ESHA are given, and the automatic model transformation is implemented. Based on our Modana tool and model verification tool UPPAAL-SMC, we present the implementation of the proposed method.

Finally, we study a specific case -- intelligent temperature control system with our proposed method. 
%With the help of domain ontology, the design model of intelligent temperature control system is built based on extended MARTE/UML language. Then the model is transformed to ESHA model automatically, and energy consumption is verified and evaluated with the UPPAAL-SMC tool.
The experimental results show that the method not only supports the complete modeling of smart buildings, but also can analyze system energy consumption and other properties with stochastic model checking technology.

{\sihao{\textbf{Keywords:}}} \textit{Smart Building, Domain Ontology, MARTE, 
Stochastic Hybrid Automata, Stochastic Model Checking.}


































