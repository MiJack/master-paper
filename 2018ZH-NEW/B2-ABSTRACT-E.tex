\newpage

\addcontentsline{toc}{chapter}{ABSTRACT}
\vspace{-500pt}

\chapter*{\zihao{2}\heiti{ABSTRACT}}


%伴随着移动应用迅猛的发展,研究人员开始关注如何分析移动应用的业务逻辑,了解应用的运行行为。
%不同于传统应用程序,Android应用程序采用的是基于事件驱动的系统架构和面向组件的编程模式,业务逻辑对回调函数和多线程交互的依赖性高。
%上述特性使得程序的业务逻辑分散在不同的代码片段(例如方法、线程、组件等)中,阻碍了应用程序执行过程的建模,对了解Android应用程序执行细节造成了挑战。


With the rapid development of mobile apps, researchers have focused their attention on  mobile apps analysis.
They want to know the runtime behavior for the application via technical approaches.
Unlike traditional programs, Android apps, based on  event-driven architecture,  take the components like Activity as  essential building blocks and  rely on callback functions and multithreaded communication heavily.
These Android features split the business logic into diffrent segments, i.e., methods, threads, and components,  making trouble to the modeling for application execution , and  posing the challenge to understand the execution for Android apps.



%为此,本文提出并实现了一个可用于生成Android应用程序动态函数调用图的系统——RunDroid。
%RunDroid使用源程序代码插桩获取用户方法的执行信息,通过运行时方法拦截获取获取系统方法的执行信息,将这些信息以日志的形式保留下来。
%根据应用程序运行期间产生的日志信息,RunDroid能还原出应用的动态函数调用图。
%RunDroid构建的函数调用图不仅反映了函数间的调用关系,还能反映方法对象关系和方法间的触发关系,体现Android中Activity组件的生命周期,为程序分析工作提供必要的运行时信息。


This paper proposed  RunDroid, a system which can be used to build the dynamic function call graphs for Android apps.
RunDroid obtains  the execution information of user-level method via  the source-code level instrumentation,
gains the information of system-level method by runtime intercept.
Using the log information, generated while app running, RunDroid can recover the dynamic function call graph for the application.
The call graph, constructed by RunDroid, not only shows the calling relationship, but also reflects the method-object relationship, the trigger relationship between the methods, and the lifecycle of  Activity component, 
and provides the important runtime details for program analysis.





In this paper, we download 9 Android apps from the open source community F-Droid and use RunDroid to build dynamic function call graphs.
Results show that the trigger relationship is ubiquitous during app running and plays the major role in the Android.
Also, we compare the call graphs generated by RunDroid and FlowDroid.
Compared with the FlowDroid's static call graph, the dynamic call graph generated by RunDroid can accurately display the execution detail for the apps, express the calling relationship and trigger relationship between functions, and restore the lifecycle of the Android component Activity.
Finally, we apply RunDroid to fault localization for exploratory experiments.
The result shows that the call graph provided by RunDroid can  reflect the causal relationship of function execution comprehensively.
Compared with the previous technical, RunDroid makes the causal relationship model between methods more complete, and computes more program dependency information, which helps to improve the accuracy of fault location result.






{\sihao{\textbf{\newline Keywords:}}} \textit{Android, Call Graph, Dynamic Analysis, Lifecycle, Multi-Thread}


































