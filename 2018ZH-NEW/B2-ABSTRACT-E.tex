\newpage
\vspace{-5.5cm}
\chapter*{\zihao{2}\heiti{ABSTRACT}}
\vspace{-0.5cm}

In Android, the logic of the application is spread across different code segments (such as methods, threads, components, etc.), which makes some static analysis tools less accurate when analyzing.
To help researchers and developers understand the execution of Android applications, we have provided RunDroid, a tool for restoring dynamic call graphs for Android application runtimes, to help assist static tools to provide more accurate program analysis results.
RunDroid uses the combination of source code instrumentation and runtime method interception to capture the application execution information of the application at the application layer and system level, and restore the call relationship between methods.
On this basis, RunDroid uses the dependencies between objects and methods to further restore the trigger relationship between the methods (for example), and display the Android features in the running graph in the call graph.

In addition, we also compare RunDroid with static analysis tools to analyze the advantages and disadvantages of the two techniques in generating function call graphs.



{\sihao{\textbf{Keywords:}}} \textit{Android, Call Graph, Dynamic Analysis, \question{Multi-Thread}}


































