\newpage

\addcontentsline{toc}{chapter}{ABSTRACT}


\chapter*{\zihao{2}\heiti{ABSTRACT}}




With the rapid development of mobile apps, researchers have focused their attention to the mobile apps analysis.
Usually, the function call graph, generated via technique , can help people to understand the runtime behavior for the application.
Unlike traditional programs, Android apps, based on the  event-driven architecture,  takes the components like Activity as  the essential building blocks, and  rely on the callback functions and multithreaded communication heavily.
These Android features split the business logic into diffrent segments, i.e., methods, threads, and components,  and make trouble to the modeling for application execution process , and cause some troubles to understand the execution for Android apps.



This paper proposed  RunDroid, a technical that can be used to generate the dynamic function call graphs for Android apps.
RunDroid combines two technicals, source-code-level instrumentation and runtime method interception, to log the execution information for the method in app.
Using the log information generated while app running, RunDroid can recover the dynamic function call graph of the application.
The call graph, constructed by RunDroid, not only shows the calling relationship, but also reflects the method-object relationship, the trigger relationship between the methods, and the lifecycle of  Activity component.


In this paper, we download 9 android apps from the open source community F-Droid and use RunDroid to build the dynamic function call graphs.
The results show that the trigger relationship is ubiquitous during app running and plays the major role in the Android business logic.
Also, we compare the call graphs generated by RunDroid and FlowDroid.
Compared with the FlowDroid's static call graph, the dynamic call graph generated by RunDroid can accurately display the execution detail for the apps, express the calling relationship and trigger relationship between functions, and restore the lifecycle of the Android component Activity.
Finally, we also apply RunDroid to the fault localization for exploratory experiments.
The result shows that the call graph provided by RunDroid can  reflect the causal relationship of function execution comprehensively.
Compared with the previous technical, RunDroid makes the causal relationship model between methods more complete, and computes more program dependency information, which helps to improve the accuracy of experimental results.





{\sihao{\textbf{Keywords:}}} \textit{Android, Call Graph, Dynamic Analysis, Lifecycle, Multi-Thread}


































