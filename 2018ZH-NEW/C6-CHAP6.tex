\chapter{总结与展望}
\label{ch6}
\section{总结}
	智能建筑是一种典型的能量信息物理融合系统,除了密切关注系统能耗之外,还具有混成和随机的特性。对于这一类复杂系统,使用模型驱动的系统开发方法可以有效控制系统的复
杂度、提高系统质量和系统开发复用性。针对智能建筑的建模与验证问题,本文给出了利用领域本体指导建模系统的扩展MARTE/UML模型、模型转换为能耗随机混成自动机并实现模型验证与分析的集成方法。具体来说,本文的主要工作如下:
	\begin{itemize}
	\item 由于智能建筑等CPES的复杂性,很难明确系统中的概念结构以及概念间的关系,针对此问题,本文给出了智能建筑的领域本体,定义了系统中的基本术语、概念间的关系以及特定的约束和规则。利用智能建筑领域本体,可以辅助构建系统的初步设计模型;
	\item 现有的MARTE/UML模型在对CPES建模时,存在一定的局限性,为了解决这一问题,本文定义了CPES中广泛存在的两种随机行为,扩展了MARTE的元素,并给出了UML类图、顺序图和状态图的扩展定义;
	\item 虽然MARTE/UML模型能够实现系统的多视角建模,但由于其半形式化的特性,无法对模型进行验证。为此,本文扩展了随机混成自动机模型,使其能够显式建模系统中的能耗,并给出了扩展的MARTE/UML模型到能耗随机混成自动机的映射规则和转换算法,以实现模型自动转换。最终得到的系统能量随机混成自动机模型可以利用现有的统计模型检测算法进行系统的定性、定量验证,从而分析系统的能耗等性质。基于我们研发的Modana平台和模型检测工具UPPAAL-SMC,我们给出了本文所提的建模、分析方法的实现框架;
	\item 为了验证本文所提的智能建筑能耗建模与分析方法的可行性和有效性,我们以智能温控系统为例,实现了领域本体指导建模初步模型、模型设计和模型验证的全过程,并通过实验数据分析了系统能耗与不同的调度策略和用户行为模式的关系。
	\end{itemize}
	
\section{下一步工作}
	本文基于领域本体、MARTE/UML和随机混成自动机理论,提出了智能建筑能耗建模与分析方法,并基于Modana平台和UPPAAL-SMC工具给出了所提方法的实现框架。关于利用智能建筑领域本体指导建模系统的初步模型,可考虑设计更为用户友好的模型设计界面,提供不同的设计模板以方便研究人员实现快速建模;关于智能建筑系统的能耗随机混成自动机的验证,可考虑将常见的验证属性设计为特定的属性模板,展现在工具中,以支持对于常见的系统能耗属性的验证、分析。此外,目前的ESHA模型是借助UPPAAL-SMC工具实现建模的,通过自定义特定的变量来建模系统的能耗,下一步可考虑在Modana中开发专属的ESHA模型建模模块,定义能量这种特殊的变量类型。