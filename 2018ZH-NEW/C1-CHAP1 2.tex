\chapter{绪\hskip 0.4cm 论}
\label{chap1}

\section{研究背景}

随着城市智能化进程的推进,感知技术和泛化计算逐渐渗透到城市的方方面面,大数据和基于这些数据的技术研究也应运而生。位置感知设备如智能手机、车载GPS等的普及和应用,产生的大规模移动轨迹数据成为基于位置服务相关研究的重要数据源。城市中道路上每天行驶的出租车,正如城市交通的移动传感器,这些出租车装配的GPS设备,每天能够收集到大量的移动位置序列信息和车载状态信息,这些数据蕴含着丰富的交通信息和用户行为信息,通过对轨迹数据进行分析与挖掘,我们能够了解交通状况,合理规划行程,发现人群行为特征,协助改善交通状况等。

出租车轨迹数据是由一系列包含GPS时间、经度、维度、出租车状态等信息的数据点组成。通常为了安全保障和管理调度,出租车会以一定的时间间隔向数据中心上传相应的地理位置信息和状态信息,从而积累了大量的出租车的轨迹点数据,这些数据能够从一定程度上反映了城市交通状况和人们的出行情况,具有很大的科研和应用价值。图\ref{fig:1000taxi_pdf}中是由1000辆上海市出租车在两个小时的时间段内收集到的轨迹数据点绘制而成,虽然仅仅是出租车轨迹数据的一个小的样本,但从中我们能够清晰的看出这些轨迹点对实时交通状况的刻画。以北京和上海为例,北京市有超过6万辆出租车,上海市则超过5万辆,假如以10s为间隔对运营中的出租车进行中高密度轨迹数据采样,每天收集到的上海市出租车轨迹数据点将超过1亿个,数据量每天以超过10GB进行增长。同时,GPS设备精度和度量误差也带来了一定程度的数据质量的下降。此外,对轨迹数据的处理和挖掘通常会结合路网数据、社交媒体数据、兴趣点数据库等不同属性的数据开展分析,以一个持续时间段的轨迹数据为基础,数据规模、数据质量和数据复杂性均对挖掘和应用工作提出了巨大的挑战。


\begin{figure}[!htb]
\centering
\includegraphics[width=15 cm]{fig/1000taxi.pdf}
\caption{1000辆上海市出租车2个小时的轨迹点数据} %\vspace*{-1.0cm}
\label{fig:1000taxi_pdf}
\end{figure}

目前基于轨迹数据的研究多以大数据来解决城市交通面临的挑战为目的,微软亚洲研究院提出了城市计算的概念,城市规模不断扩大和人口不断增加等带来的能耗、污染、环境和交通等问题逐渐凸显,通过大数据计算来解决资源合理配备等诸多问题提供更好的城市生活的需求迫在眉睫,城市计算通过城市感知、数据挖掘、智能提取和服务四个主要环节来建立一个生态循环系统,以产生一个双赢的结果,即为城市人民提供更美好的城市生活,同时也让城市变得绿色智能,提升整个城市环境\cite{DBLP:conf/huc/ZhengLYX11}。很多基于城市轨迹数据和基于位置服务的研究工作都与城市计算的愿景不谋而合,如从大量的轨迹数据中发现最频繁的路径从而进行更合理的路径规划\cite{DBLP:conf/sigmod/LuoT0N13}\cite{DBLP:conf/icde/SuZHJ0Z14},利用出租车轨迹数据预测并掌握交通状况,为城市交通规划提供参考\cite{DBLP:conf/pervasive/CastroZL12},也为出租车司机提供更加节能高效的驾驶路线\cite{DBLP:conf/kdd/QuZLLX14}\cite{song37},挖掘异常轨迹点发现道路变化或不良出租车驾驶行为\cite{DBLP:conf/huc/ZhangLZCSL11},预测交通异常状况\cite{DBLP:journals/dke/PangCLZ13},另外也有结合社交网络中的位置信息进行位置推荐或者交通事件预测\cite{export:191797}\cite{export:201131}等等。

出租车轨迹能够全方位覆盖城市路网交通,既能反映出实时的交通密集度和流通度,也能反映出人群的出行规律和区域特征。利用蕴含巨大知识价值的出租车轨迹数据,探索高效的数据挖掘和智能推荐算法,能够建立有效地智能交通感知网络,协助用户智能出行,发现城市新热点和潜在问题,并辅助解决城市中相关的其他问题。但同时也面临诸多问题,如(1)轨迹数据规模较大,传统的分析方法不能提供高效的解决方案,实践中通常需要定制计算逻辑;(2)轨迹数据具有很强的时间和空间特性,利用这些特点能够关联更多的异构数据,实现更加准确的分析,但是也增加了数据查询和处理的难度,增大了数据分析的维度;(3)数据稀疏性,即使轨迹数据具有一定规模,数据稀疏问题也仍然不可避免,例如:出租车覆盖的路段集中在市中心,而某些区域出现较少轨迹数据,或者对于一个出租车轨迹而言,如果在采样率较低的情况下,两个GPS数据点之间间隔多个路段,增加了轨迹数据的不确定性。除此以外,对轨迹数据的查询、处理、挖掘分析等多个方面仍然面临各种挑战,本文工作着眼于利用出租车轨迹来解决人们出行中的打车难问题,主要研究轨迹数据的分布式处理框架和打车推荐服务中涉及的聚类挖掘、时间预测等关键问题。


\section{本文工作与主要贡献}

本文以海量出租车轨迹数据为研究对象,基于已有研究成果,以智能打车推荐为应用目标,建立对轨迹数据的分布式处理框架和挖掘分析系统,并实现在线的查询与推荐服务。解决的问题包括:轨迹预处理、轨迹数据聚类、轨迹数据查询、预测和推荐模型建立等多个方面。
本文主要的研究工作内容如下:

\paragraph{对轨迹数据的分布式处理} 

在对轨迹数据进行挖掘和分析的之前,数据的预处理工作能够提高模型准确度并辅助模型抽取出所需数据,通常包括对数据的降噪处理和过滤、轨迹数据到路网的映射等。鉴于轨迹数据的数据规模,并行的数据处理策略能够大大提高对批量历史数据处理的效率,本文工作中建立基于Map-Reduce的通用轨迹处理框架,实现在不同采样密度下优化的路网匹配算法,并将分布式处理框架应用于数据过滤、路网匹配、特征抽取等轨迹数据处理的多个关键阶段。

\paragraph{兴趣点和兴趣区域挖掘}

兴趣点和兴趣区域通常作为推荐元素向用户推荐,在不同的挖掘任务中,根据推荐的目标不同采用的方法也不同,聚类是发现轨迹数据特征的最常用方法之一,而对于时空特性明显的地理位置数据,聚类算法的设计、度量方法的选择、数据查询结构等均是该部分的主要研究内容。本文以打车推荐为目的,重点讨论采用基于密度的聚类方法对候选打车点和热门目的地的挖掘方法。

\paragraph{打车推荐服务应用和候车时间预测}

基于位置的服务泛指一类利用定位技术获得当前位置信息,再通过无线网络得到某项服务的技术\cite{ayzhou_lba},并且能够为大量普通用户提供服务。利用历史出租车轨迹数据,我们可以为用户提供智能出行建议,减少用户行程中浪费不必要的时间。本文基于多种模型对不同路段出租车空车到达时间进行建模,利用兴趣点挖掘技术提供备选打车点和目的地方案,建立预测准确、推荐合理的打车点推荐系统,并提供查询应用服务。


\paragraph{本文主要贡献如下:}

\begin{itemize}
	\item 建立了基于Map-Reduce的分布式轨迹处理框架,并在此框架下实现了轨迹预处理和优化的路网匹配算法。
	\item 采用基于密度的聚类方法发现兴趣点和兴趣区域,从而找到备选打车点和热门目的地。
	\item 实现了打车智能推荐系统,该包含了数据预处理、路网匹配、特征抽取、路段聚类、在线预测、查询推荐等多个模块,完成了基于泊松过程、逻辑回归等的出租车等待时间预测算法。
	\item 提供在线打车查询和推荐应用,能够支持多种客户端实时的请求相应。
\end{itemize}


\section{组织结构}

第2章中介绍与本文工作相关的研究进展;第3章中介绍了本文实现的分布式轨迹数据处理框架,并具体阐述了本文采用的路网匹配方法和实验效果;为了实现后续的基于位置的打车推荐服务,第4章中介绍了利用轨迹点的聚类方法进行兴趣点和兴趣区域的挖掘,包括关键的辅助索引结构和优化的基于密度的聚类算法;第5章基于前面的轨迹数据和聚类结果,实现了对打车点和目的地的推荐应用,其中,重点介绍了对打车点候车时间的预测模型,以及在线的查询和推荐算法。第6章总结全文,对后续的研究工作进行展望。



