\chapter{RunDroid的系统实现 }
\label{chp:implement}


\foreach \t  in {1,...,3}{\todo{缺点东西,待补充 \t} \newline }
\section{相关技术介绍}


\subsection{srcML}


srcML~\cite{collard2013srcml}是轻量级源代码分析工具,它可以将程序的抽象语法树以XML的形式展现给用户,支持C、C++、C\#以及Java等多个语言的语法解析。
通过srcML的语法解析器\eat{采用的是基于LL(*)算法实现的语法解析器 ANTLR 2.7.7,}
解析得到的XML内容保留了源代码中完整的语法树信息。
\eat{
同时,srcML还提供了一个强大工具集,支持对生成内容的查询、分析以及修改,可以用于架构设计、语言研究、软件重构等场景,应用于软件工程、编程语言、并行和分布式处理等多个领域。
}
在RunDroid,srcML作为RunDroid预处理器中的重要组成部分,承担源代码语法解析的主要职责,辅助完成用户方法层面的日志代码的编织。
\subsection{Xposed框架}


%xposed是一个模块框架,可以在不接触任何APKs的情况下改变系统和应用程序的行为。这很棒,因为这意味着模块可以在不做任何改变的情况下为不同版本甚至是rom工作(只要原始代码没有太多改变)。撤销也很容易。由于所有更改都在内存中完成,您只需关闭模块并重新启动即可恢复原始系统。还有许多其他优点,但这里还有一个:多个模块可以对系统或应用程序的同一部分进行更改。有了改进的APKs,你可以选择一个。除非作者用不同的组合构建多个APKs,否则无法将它们组合起来。


Xposed~\cite{Xposed}是一个基于Android系统的运行时行为修改框架。
在不修改程序源代码的情况下,基于Xposed开发的第三方插件通过将目标方法关联到函数执行回调上,进行方法参数和返回值的重写和额外方法逻辑的添加等操作,达到目标方法行为修改的目的。
相比定制化Android系统,Xposed同时兼容大部分Android系统版本,实现成本低。
\eat{
Xposed的实现原理具体如下:由于Android系统的所有的应用程序进程都是由Zygote进程孵化而来,Xposed通过替换程序\code{/system/bin/app\_process},使得系统在启动过程中加载Xposed的相关文件,将所有的目标方法指向Native方法xposedCallHandler,维护目标方法和对应的钩子方法(Hook Function)的映射关系,从而实现对Zygote进程及Dalvik虚拟机的劫持;
当程序执行到目标方法时,xposedCallHandler会完成目标方法的原有代码和对应钩子方法的调度,达到对目标方法劫持的目的。
}
利用Xposed提供的类似面向切面编程的API接口,RunDroid中的运行时拦截器可以对任意方法(包括用户方法和系统方法)执行过程进行动态拦截,达到系统方法信息的日志记录的目的。


\subsection{Neo4j}

Neo4j~\cite{Neo4jthe19}是基于Java语言开发的图数据库,可以用于存储图结构相关的数据结构。
%与传统的基于关系模型的存储结构不同,Neo4j的存储模型是基于图论开发的,遵循属性图数据模型。
Neo4j的数据主要分为节点(Node)和关系(Relationship)两大类,分别对应图论中的点与边。
Neo4j支持在关系和节点上添加自定义键值属性,为节点指定标签,为关系指定类型等等。
在数据操作接口方面,Neo4j支持Cypher、Java API和RESTful API等多种方式,同时也提供了友好的数据浏览界面用于数据的展示与修改。
\eat{由于基于属性图数据模型,Neo4j通常适用于和图关系有着密切关系的应用场景:例如社交网络分析,公共交通网络研究以及地图网络拓扑等场景。}
在RunDroid,Neo4j是调用图构建器的主要组成部分,主要承担着拓展函数调用图的存储、查询、展示的职责。


\section{模块实现}



\subsection{预处理器}

在\autoref{chp:design}中,我们提到预处理器在RunDroid中的作用主要是日志代码编织。
在现有的技术上,代码编织主要分为两类:基于源代码的代码编织和基于字节码的代码编织。
通常采用的是字节码编织技术,例如Aspect、Emma等。
但是,这种字节码插桩技术在运行过程中会生成过多的方法数,在Android应用构建过程中可能存在方法数65K限制问题。
为此,预处理器采用的方案是基于源代码的代码编织方案。
该方案通过修改源程序,将日志记录代码直接写入在方法体内部,避免了新方法的引入,在一定程度上避免了方法数65K限制问题。
%   进而通过日志记录器输出相应的日志。


% 拦截器组件基于Xposed框架[5],它拦截在应用程序层和Android框架之间传递的消息。
% 拦截器记录在两个层之间进行的每个方法调用,并将它们与应用程序层中对应的方法调用相关联,以产生完整的方法调用跟踪。
% RunDroid维护感兴趣的方法列表,如生命周期方法和隐式回调,以便日志文件包含每次执行期间调用的方法调用


\subsection{日志记录器}

日志记录器的职责是将方法执行的消息以文件的形式持久化下来。
针对不同的方法类型(静态方法与非静态方法、用户方法与系统方法等),日志记录器提供了不同的API,帮助我们记录相应方法执行的日志信息。
在日志内容上,日志记录器还会记录每个方法执行的所处线程、方法签名标识、所处阶段(方法开始执行阶段/方法执行完毕阶段)以及相关的方法对象信息。
方法对象信息主要包括对象的类型、属性和全局ID(通过Java API\code{System.identityHashCode(Object)}获取)等。
同时,日志记录器还支持对象信息的自定义:开发人员可以根据自身的需要为不同类型的对象输出不同对象数据信息。
% 在底层实现上,考虑日志读入文件对程序执行效率的影响,我们评估了多种日志写入方式,最终我们发现基于mmap的日志记录方式效率最高,对程序运行的影响最小。



\subsection{调用图构建器的实现}

调用图构建器的主要功能是拓展函数调用图的构建、存储和展现。
在实现上,调用图构建器主要有以下几个部分组成:对象标识管理器、调用图存储器以及触发关系生成器。

调用图构建器以应用程序在运行时的日志信息作为输入,利用日志中的方法执行信息在调用图储存器中初步建立函数调用图。
同时,对象标识管理器会根据对象使用的上下文(如该和哪个方法产生关联,具体的关系是什么等)赋予相应的标识ID,并将其添加到函数调用图中。
最后,根据调用图储存器中方法和对象之间的关系结合具体的触发关系,触发关系生成器会还原出方法间的函数触发关系。
% RunDroid通过动态执行跟踪匹配这些调用来应对这个挑战,这导致开发三个子组件:引用存储库组件、映射引擎组件、合成器组件。
% Reference Repository组件构建于Neo4j[4]之上,它处理日志文件,并存储捕获的方法及其执行调用。
% Mapping Engine组件构建于soot[13]之上,从日志历史中识别异步调用对,搜索它们相应的实例化类,并将这些捕获的多线程方法调用存储为“触发器”关系。
% 然后,Synthesizer组件将这些触发器关系与捕获的执行调用集成到一个完整的执行调用跟踪中,并存储在Reference Repository中。
在技术选型上,调用图存储器采用Neo4j作为调用图的底层存储引擎,触发关系生成器中的规则定义由Cypher脚本和Soot共同实现。

\section{技术实现细节}


\subsection{对象的标识}
\subsubsection{Message对象的标识?}
\subsection{系统方法的捕获}

在实现上,运行时拦截器是基于Xposed的插件,它维护的列表包括了所有需要拦截的系统方法(下称为目标方法)。%,如\autoref{tbl:hookMethodList} 所示。
每当应用程序启动时,运行时拦截器通过Xposed提供的API\code{ XposedHelpers.findAndHookMethod()}将\autoref{tbl:hookMethodList}中的目标方法绑定到方法钩子(即\eat{Xposed提供的XC\_MethodHook的子类 ,}类HookCallBack)上。
在应用程序的运行过程中,HookCallBack对象的\code{beforeHookedMethod(MethodHookParam)}会在目标方法执行之前被调用,
\code{afterHookedMethod(MethodHookParam)}会在目标方法执行后调用。
最终,HookCallBack对应会将方法执行的相关信息传递给日志记录器。
\foreach \t  in {1,...,22}{\newline\t }

 \section{本章小结}

本章介绍了RunDroid的实现。
本章先对RunDroid使用到的技术(srcML、Xposed、Neo4J)做了基础的介绍。
并在从技术基础上,我们介绍了RunDroid中各模块的具体实现。
