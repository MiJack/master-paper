\chapter*{ 致\qquad 谢}

\addcontentsline{toc}{chapter}{致  谢}


光阴似箭,岁月如梭,转眼间,研究生生涯即将划上句号。
回首过往,颇是感慨。
期间受到不少老师、同学、家人的支持、帮助和鼓励,才得以走到今天。

首先,我想感谢我的导师徐立华老师。
徐老师不仅在学术上精心指导,同时还给予了我充足的发展空间。
这也使得我可以充分地思考问题,并深入探索尝试,并有所收获。
我所有的收获和成就都离不开您的贡献和努力。
在此谨向徐老师致以诚挚的谢意和崇高的敬意。




在研究生学习期间,我还受到了很多老师的指导和鼓励。
感谢谢瑾奎老师、徐飞老师、窦亮老师、肖旭生老师。
谢瑾奎老师学识渊博,循循善诱,使我从中学习到不少思考问题的新角度、探索事物的新思路。
徐飞老师在组会和课堂上时给予的指导,行为举止间展现出学者的严谨。
窦亮老师给予我助教的机会,使我学会了换位思考。
当然,还要感谢凯斯西储大学肖旭生老师。
和肖老师的交流使我学到了很多思考和解决问题的方法。
同时,我还要感谢年级辅导员杨文彧老师。
感谢各位老师的指导和关照,你们的言传身教使我收益颇丰。

此外,我还想感谢陈森师兄、范玲玲师姐、谷林涛师兄、杨帅师兄在学术上给予的帮助。
同时,也要感谢卜文奇、张雨、蒋欢、吴瑜珠、马俊奇、唐崇斌、汪庆顺、刘剑、叶莎莎、龚鑫等同学。
和你们共同学习、生活的时光非常开心,感谢各位在学习和生活上给我带来的帮助与照顾。


最后,我还要感谢我的父母和家人,感谢读研期间你们在经济和精神上对我的支持和鼓励。
正是因为你们的支持,我才坚持走完硕士学习生涯,同时也让我对生活的意义有了更深的理解和感悟。

\vspace{0.2cm} 

\hspace{10.6cm}  二〇一八年九月 


