\addcontentsline{toc}{chapter}{摘要}

\chapter*{\zihao{2}\heiti{摘~~~~要}}


\setlength{\baselineskip}{25pt}	



伴随着移动应用迅猛的发展,研究人员开始关注如何分析移动应用的业务逻辑,了解应用的运行行为。
不同于传统应用程序,Android应用程序采用的是基于事件驱动的系统架构和面向组件的编程模式,业务逻辑对回调函数和多线程交互的依赖性高。
上述特性使得程序的业务逻辑分散在不同的代码片段(例如方法、线程、组件等)中,阻碍了应用程序执行过程的建模,对了解Android应用程序执行细节造成了挑战。



为此,本文提出并实现了一个可用于生成Android应用程序动态函数调用图的系统——RunDroid。
RunDroid使用源程序代码插桩获取用户方法的执行信息,通过运行时方法拦截获取获取系统方法的执行信息,并将信息以日志的形式保留下来。
根据应用程序运行期间产生的日志信息,RunDroid能还原出应用的动态函数调用图。
RunDroid构建的函数调用图不仅可反映函数间的调用关系,还能反映方法对象关系和方法间的触发关系,体现Android中Activity组件的生命周期,为程序分析工作提供必要的运行时信息。




在本文中,我们从开源社区F-Droid中下载9个开源应用程序,利用RunDroid产生动态函数调用图,统计调用图中的函数关系数量。
实验结果显示函数触发关系在应用程序的执行过程中普遍存在,并且在业务逻辑中承担着主要的作用。
同时,我们还将RunDroid产生的动态函数调用图和FlowDroid产生的静态函数调用图进行对比分析。
相比FlowDroid,RunDroid产生的函数调用图能够准确地反映应用程序的执行过程,表现函数间的调用关系和触发关系,真实地还原Android Activity组件的生命周期变迁。
最后,我们还将RunDroid和错误定位相结合,进行了探索性的实验。
实验结果显示,RunDroid提供的函数调用图全面地反映了函数执行的因果关系。
相比已有技术方案,RunDroid能够使得方法间的因果关系模型更完整,体现更多的程序依赖信息,有助于提升错误定位的准确度。




\sihao{\heiti{ 关键词:}} \xiaosi{Android,函数调用图,动态分析技术,生命周期,多线程}
