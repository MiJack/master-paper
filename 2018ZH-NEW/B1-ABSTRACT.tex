\vspace{-2.5cm}
\chapter*{\zihao{2}\heiti{摘~~~~要}}
\vspace{-1cm}

\setlength{\baselineskip}{25pt}	
	%能量信息物理融合系统结合了信息技术和物理设施来对能量进行有效管理和控制。
	%通过计算、通信和控制技术的深度协作,能量信息物理融合系统(Cyber-Physical Energy System,CPES)实现了网络基础设施和物理基础设施的有机融合,且利用智能控制技术实现了系统能量的有效管理。由于CPES存在于复杂的物理环境中,与外界的各种随机行为存在交互,且系统中连续行为和离散行为共存,因此,CPES具有随机和混成的特点。
	能量信息物理融合系统(Cyber-Physical Energy System,CPES)是一类具有随机和混成特点的复杂系统。
	智能建筑(Smart Building)是一种典型的CPES,它通过对系统中各种功能构件的智能控制来改变室内环境参数,从而为人类提供舒适的生存环境,并且力求减少系统的总体能耗。
	
	传统的系统开发方法在设计、开发完成后才进行测试,测试完成后再对系统进行修改往往成本过高且耗时过长。为了在开发过程的早期发现系统设计的错误与缺陷,模型驱动开发(Model-Driven Development, MDD)方式具有重要意义。
	%MDD以模型构建为中心,能够借助模型转换技术,从系统的需求模型过渡到可执行的系统模型,并可利用模型检测技术验证、分析可执行模型的属性,从而保证模型的质量。
	然而,使用MDD方法对智能建筑这一类复杂的CPES建模面临的挑战是:
	1)针对CPES的特性,结合智能建筑领域的特征,如何有效地重用该领域的知识,指导构建系统的设计模型?
	2)如何扩展标准建模语言MARTE/UML以构建系统的设计模型?
	3)如何建模CPES的可执行模型,并实现关于系统能耗等性质的验证与分析?
	%一个完整的系统建模过程包括需求分析、模型设计,以及模型验证。
	%1)鉴于CPES跨领域的特点,如何在诸多概念中确定与系统需求相关的概念及其之间的关系,从而明确建模的元素与边界;
	%2)目前最常用的MARTE/UML建模语言对于CPES的混成和随机特性的语义支持不够完善;
	%可执行层的自动机模型并没有明确定义对于系统能耗建模、分析的方式。
	%在此背景下,本文提供了一个面向智能建筑的,从系统需求分析、模型设计,到模型验证与分析的集成方法。
	针对以上问题,我们提出了一种智能建筑能耗的建模与分析方法,主要贡献包括以下内容:
	
	首先,基于建筑信息模型、参照本体构造方法的基本步骤,给出了智能建筑的领域本体,包括智能建筑中的基本概念、概念间的关系和约束,并指出了如何利用领域本体指导建模系统初步模型。
	
	其次,针对智能建筑等CPES的能耗感知、随机和混成特性,提出扩展的MARTE/UML建模规范:1)定义了CPES中两种典型的随机行为;2)扩展了MARTE中的数据类型和表达式以完善其建模能力;3)扩展了UML的类图、顺序图和状态图的元模型以支持对能耗、随机和混成的建模。

	虽然MARTE/UML模型能够以直观的形式实现系统的多视图建模,但由于其半形式化的特性,无法对模型的属性进行验证、评估。针对此问题,我们以能耗为中心扩展了随机混成自动机,定义了能耗随机混成自动机的语法、语义,并给出从扩展的MARTE/UML状态图到能耗随机混成自动机的映射规则,以实现模型的自动转换。基于我们的Modana工具和模型验证工具UPPAAL-SMC,可实现本文所提的智能建筑能耗建模与分析方法。
	
	最后,利用我们所提的智能建筑能耗建模与分析方法,针对一个具体案例——智能温控系统进行了研究。
	%在领域本体的指导下,基于扩展的MARTE/UML语言构建了智能温控系统的设计模型;并使用模型转换技术得到对应的ESHA模型,基于UPPAAL-SMC工具分析、评估系统的能耗。
	实验证明该方法不仅支持对智能建筑系统的完整建模,而且可以借助统计模型检测技术分析、评估系统的能耗等性质。
	%给出了从系统的需求分析、建模MARTE/UML模型、模型转换为ESHA和系统验证与分析的全过程。实验结果表明,本文提出的建模与分析方法不仅能支持对智能建筑系统的完整建模,而且可以帮助分析系统的能耗和其他因素的关系。
	

\hspace{-0.5cm}
\sihao{\heiti{关键词:}} \xiaosi{智能建筑,领域本体,MARTE,随机混成自动机,统计模型检测}
