\addcontentsline{toc}{chapter}{摘要}

\vspace{-2.5cm}
\chapter*{\zihao{2}\heiti{摘~~~~要}}
\vspace{-1cm}

\setlength{\baselineskip}{25pt}	

伴随着移动应用迅猛的发展,研究人员开始关注如何通过技术手段生成函数调用图用于移动应用的分析,了解应用的运行行为。
不同于传统应用程序,Android应用程序采用的是基于事件驱动的架构,采用面向组件的编程模式,业务逻辑对回调函数和多线程交互高度依赖。
上述特性使得程序的业务逻辑分散在不同的代码段(例如方法、线程、组件等)中,阻碍了应用程序执行过程的建模,对了解Android应用程序执行细节造成了困扰。



本文提出并实现了一个可用于生成Android应用程序动态函数调用图的技术方案——RunDroid。
RunDroid利用源程序代码插桩和运行时方法拦截的相结合的方式,将应用程序的方法执行信息以日志的形式保留下来。
根据应用程序运行期间产生的日志信息,RunDroid能还原出应用的动态函数调用图。
RunDroid构建的函数调用图不仅反映了函数间的调用关系,还能反映方法对象关系和方法间的触发关系,体现Android中Activity组件的生命周期。




在本文中,我们从开源社区F-Droid中下载9个开源应用程序,利用RunDroid产生动态函数调用图。
实验结果显示函数触发关系在应用程序的执行过程中普遍存在,并且在业务逻辑中承担着主要的作用。
同时,我们还将RunDroid产生的动态函数调用图和FlowDroid产生的静态函数调用图进行对比。
相比FlowDroid,RunDroid产生的函数调用图能够准确地反映应用程序的执行过程,表现函数间的调用关系和触发关系,真实地还原Android Activity组件的生命周期变迁。
最后,我们还将RunDroid和错误定位相结合,进行了探索性的实验。
实验结果显示,RunDroid提供的函数调用图全面地反映了函数执行的因果关系。
相比之前技术方案,RunDroid能够使得方法间的因果关系模型更完整,体现更多的程序依赖信息,有助于提升实验结果的准确度。


综上所述,RunDroid生产的函数调用图,可以准确的反映应用程序的执行过程,帮助研究人员了解应用程序的执行过程。

\hspace{-0.5cm}
\sihao{\heiti{关键词:}} \xiaosi{Android,函数调用图,动态分析技术,生命周期,多线程}
