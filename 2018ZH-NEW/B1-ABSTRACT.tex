\vspace{-2.5cm}
\chapter*{\zihao{2}\heiti{摘~~~~要}}
\vspace{-1cm}

\setlength{\baselineskip}{25pt}	

在Android,应用程序的逻辑分散在不同的代码段(例如方法、线程、组件等)中,这使得部分静态分析工具在分析时,得不到精确的结果。
为了帮助研究人员和开发人员了解Android 应用程序的执行过程,我们提供了RunDroid,一个用于还原Android应用程序运行时动态调用图的工具,进而帮助辅助静态工具提供更精确的程序分析结果。
RunDroid利用源程序代码插桩和运行时方法拦截的相结合的方式,捕获应用程序在应用层和系统层的方法执行信息,还原方法间的调用关系。
在此基础上,RunDroid利用对象和方法间的依赖关系,进一步还原方法之间的触发关系(例如),在调用图中展现运行过程中的Android特性。

另外,我们还将RunDroid和静态分析工具进行对比,分析两种技术在生成函数调用图上的优缺点。


\hspace{-0.5cm}
\sihao{\heiti{关键词:}} \xiaosi{Android,函数调用图,动态分析技术,多线程}
