

\fancypagestyle{plain}{%
	\fancyhead[LE,RO]{华东师范大学硕士专业学位论文}
	\fancyhead[RE,LO]{ 本文使用到的符号说明}
}


\addcontentsline{toc}{chapter}{本文使用到的符号说明}

\chapter*{本文使用到的符号说明}




\begin{table*}[!ht]
\centering

\begin{tabular}{|c|p{10cm}|}
\hline
符号&说明\\
\hline
$o$ & 对象$o$\\
\hline
$m$ & 方法或方法执行 $m$\\
\hline	
$o.class$ &对象$o$的类型 \\ 
\hline	
$m.class$ & 定义方法$m$的类 \\
\hline	
$m.sign$ & 方法执行$m$的方法签名 \\
\hline	
 $m.methodSign$  &方法执行$m$的方法完整签名,$m.class$和$m.sign$的组合 \\
\hline	
 $m \joinrel\xrightarrow{parameter} o_p$  &  对象$o_p$是这个方法$m$的参数  \\
\hline 
$m \joinrel\xrightarrow{return} o_r$& 对象$o_r$是这个方法$m$的返回值  \\
\hline
$m \joinrel\xrightarrow{instance} o_i$&   方法$m$是非静态方法,则方法执行时我们可以获取到关联到的this指针对象$o_i$\\
\hline
$m \to m'$ & 方法$m$在执行时调用了方法$m'$\\
\hline
%$m_0 \to  m_1 \to  \dots \to m_n \to m$ & \\
%\hline
$m \stackrel{\ast}{\to} m' $ &  方法$m$扩展调用方法$m'$\\
\hline
$m  \stackrel{\ast}{ \Rightarrow } m'$ &在系统源代码层面上, 方法$m$扩展调用方法$m'$恒成立 \\
\hline
$m \hookrightarrow m'$ & 方法$m$的执行触发了方法$m'$的执行 \\
\hline					
$m \lhook\joinrel\xrightarrow{\text{\textit{触发关系}}}  m' $& 方法$m$的执行触发了方法$m'$的执行,原因为\textit{触发关系}\\
\hline
\end{tabular}
\end{table*}





