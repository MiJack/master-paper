\chapter{名称解释}


在本节,我们将给出拓展动态函数调用图构建过程中的基本术语:方法执行、方法对象、调用关系、触发关系、函数调用图、拓展函数调用图等。

\section{概念定义}
方法执行是一个方法执行相关信息的描述,方法对象对应是和方法执行相关的对象;
调用关系和触发关系描述了方法执行之间的关系。
函数调用图为所有调用关系的集合,在函数调用图上添加方法对象以及触发关系得到拓展函数调用图。相关定义如下:

\begin{myDef}方法执行\end{myDef}
	
	
	方法执行是对方法执行过程中的相关信息的描述,完整的信息包括对应方法的完整签名、执行时所处的线程$m_t$以及相关的方法对象。
	在本文中,方法执行通常用符号$m$表示。


\begin{myDef}
	方法对象(Method Object,MO)
\end{myDef}
	和方法执行相关的对象称为方法对象,可以体现对象和执行方法的相互关系。
	在本文中,方法对象通常用符号$o$表示。
	
	
	对于一个方法执行$m$,对象和方法执行的关系有如下几种:
	\begin{itemize}
		\item 若对象$o_p$是这个方法$m$的参数,记为$o_p \stackrel{parameter}{\longrightarrow} m$,或者 $ rel(m,o_p) = parameter$;
		\item 若对象$o_r$是这个方法$m$的返回值,记为$o_r \stackrel{return}{\longrightarrow} m$,或者 $ rel(m,o_r) = return$;
		\item 若方法$m$是非静态方法,则方法执行时我们可以获取到关联到的this指针对象$o_i$,记为$o_i \stackrel{instance}{\longrightarrow} m$,或者 $ rel(m,o_i) = instance$。
	\end{itemize}


\begin{myDef}
	调用关系(Invoke)
\end{myDef}
	对于程序$P$的两个方法$m_1$和$m_2$,方法$m_1$调用了方法$m_2$,则记作$m_1 \to m_2$,称为方法$m_1$调用方法$m_2$。

\begin{equation}
m_0 \to m_1 \to \dots m_n \to m  \label{equ:extend_invoke}
\end{equation}

在此基础上,对于方法$m$,若存在方法$m_i$($i=0,\dots,n , n \geqslant 0$),使得~\autoref{equ:extend_invoke}成立,则记作$m_0 \stackrel{\ast}{\to} m$,称为方法$m_0$扩展调用方法$m_n$。
特殊的,对于方法$m_1$和方法$m_2$,若$m_1 \to m_2$,则$m_1  \stackrel{\ast}{\to}  m_2$也成立。

\begin{myDef}
	触发关系(Trigger)
\end{myDef}
	
	%如果对于动态函数调用图$DCG$中两个方法(不妨记为$m_a$和$m_b$,$m_a \in DCG$,$m_b \in DCG $),
	若方法$m_a$和方法$m_b$之间同时需要满足以下三个条件,
	则两个方法存在触发关系,记为$m_a \hookrightarrow m_b$,称为$m_a$触发调用$m_b$:
	
	\begin{itemize}
		\item 方法$m_a$的执行时间总是在方法$m_b$的执行时间之前;
		\item $m_a \stackrel{\ast}{\to} m_b $不成立;
		\item $m_a$、$m_b$之间存在着一定的因果关系,包括但不限于生命周期事件,UI交互事件或多线程通信等。
	\end{itemize}


\begin{myDef}
	函数调用图(CallGraph,CG)
\end{myDef}	
	函数调用图是对程序运行时行为的描述,用有向图$CG = ( V , E)$表示。 图中的点$ v \in V $表示一个\textbf{方法执行} $m$;
	如果方法$m_1$调用方法$m_2$(即$m_1 \to m_2$),则有向边 $e = (m_1 ,m_2)$属于集合 $E$。 


\textbf{注意:}
在应用执行过程中,方法A被调用了两次,方法A的每次调用都调用了方法B,则对应的函数调用图$CG$如~\autoref{equ:dcg_sample}所示。
在调用图$CG$中,$m_a$ 和 $m_b$ 各有两个,分别对应的两次\textbf{方法执行}。
$(m_{a_{(1)}} \to m_{b_{(1)}})$对应的是第一次函数A调用函数B,
$(m_{a_{(2)}} \to m_{b_{(2)})}$对应的是第二次函数A调用函数B,

\begin{equation}
\begin{aligned}
CG = &(V,E) ,\\ 
V = & \{m_{a_{(1)}},m_{b_{(1)}},m_{a_{(2)}},m_{b_{(2)}}\}, \\ 
E = & \{  
(  m_{a_{(1)}} \to m_{b_{(1)}}) ,( m_{a_{(2)}} \to m_{b_{(2)}})
\} 
\end{aligned}
\label{equ:dcg_sample} 
\end{equation}



\begin{myDef}
	拓展函数调用图(Extended Dynamic CallGraph,EDCG)
\end{myDef}
	在函数调用图(DCG)的基础上,添加了方法对象和函数间的触发关系。
	拓展函数调用图中的节点包括方法执行节点和方法对象节点。图中的边包括描述方法间关系的边和描述方法和对象间的边:
	前者的方法间关系包括调用关系和触发关系;而后者的关系包括和方法对象相关的三个关系。
	具体定义如~\autoref{equ:def_edcg}所示:
	
	\begin{equation}
	\begin{aligned}
	EDCG =              & (V_{EDCG},E_{EDCG}) ,\\ 
	DCG =                & (V_{DCG},E_{DCG}) ,\\ 
	V_{EDCG} =      & V_{method} \bigcup V_{object} ,\\
	V_{method} =   & V_{DCG}, \\ 
	G_{EDCG} =      & G_{method} \bigcup G_{object} , \\
	G_{method} =  & E_{DCG} \bigcup \{ (m_1 , m_2) \mid m_1 \hookrightarrow m_2 \}
	\end{aligned}
	\label{equ:def_edcg} 
	\end{equation}
	
	
\section{补点内容}
\foreach \n in {1,...,9}{补点内容吧,大佬 \n \newline}


\section{本章小结}
