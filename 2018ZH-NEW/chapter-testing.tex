\chapter{系统测试}  
\label{chp:testing}


\todo{
上一章详细介绍了xxxx,其中包含xxx、xxx各功能模块。
为了xxxx,本章节中,通过xxx来展示、评估系统。
}
实验相关配置的环境如下:
实验平台是ThinkPad E430,CPU为Intel 酷睿i5 3210M,内存为12GB。
手机型号为小米 MI 5,处理器型号为骁龙820,RAM为3GB,系统为Android 6.0,内置Xposed运行环境。

\eat{


\subsection{构建效率对比实验}
为了验证源代码插桩前后对应用程序的影响,我们从开源社区F-Droid\footnote{https://f-droid.org}中随机选取了\todo{十余}个Android App,比较源代码插桩前后应用构建的耗时时间,
以评估源代码插桩对应用构建的影响。

对于每个应用,我们采用Gradle进行APK构建,通过源代码插桩前后的APK大小变化、方法数变化以及APK构建整体耗时等指标。最终我们得到的结果如 \autoref{tbl:buildResult} 所示。
从\autoref{tbl:buildResult}中,我们发现,源代码插桩前后得到的APK文件方法数变化幅度不大,基本对应的数量变化在10个左右,APK大小变化在\todo{一个百分比范围},构建时间的变化大都在\todo{一个百分比范围}上下。

由于RunDroid采用的源代码插桩方案,通过分析应用程序源代码中各个方法的语法结构,辨识出方法执行过程的方法入口和所有的方法出口,并在这些位置上提交日志代码插桩。
整个过程中,RunDroid引入了有限数量的方法,使得应用的构建耗时无明显的增加。
而文献\cite{van2013dynamic}基于AspectJ的字节码插桩方案,对于每一个方法均会在APK中声明多个新的方法(即AOP中的before通知、after通知等),以保证方法执行前后可以输出相应的信息。
该方法最大的劣势在于对于一些大型的Android应用,该方案引入的大量方法,很容易突破Android系统的方法数限制(即65536),使得应用整体的构建不可进行,APK无法构建成功。
而本方案在这个方面并没有太大的限制。


\begin{table*}[!ht]
	\centering
	
	\caption{构建效率对比实验}
	
	\label{tbl:buildResult}
	
	
	\resizebox{\textwidth}{!}{
		\begin{threeparttable}[b]
			
			\begin{tabular}{|c|c|c|c|c|c|c|c|}
				\hline
				
				\multirow{2}{0.03\linewidth}{序号}	&\multirow{2}{0.2\linewidth}{\centering{应用名}}& \multicolumn{3}{c}{未采用插桩方案}&  \multicolumn{3}{|c|}{采用插桩方案 }\\
				\cline{3-8}
				
				&	& 方法数 & APK大小 &	构建时间 &方法数 & APK大小 & 构建时间\\
				\hline
				
				1&	2   &方法数 & APK大小 &	构建时间 & 方法数 & APK大小 & 构建时间\\
				\hline
				2&	2   &方法数 & APK大小 &	构建时间 & 方法数 & APK大小 & 构建时间\\
				\hline
				3&	2    &方法数 & APK大小 &	构建时间 & 方法数 & APK大小 & 构建时间\\
				\hline
				4&	2    &方法数 & APK大小 &	构建时间 & 方法数 & APK大小 & 构建时间\\
				\hline
				5&	2    &方法数 & APK大小 &	构建时间 & 方法数 & APK大小 & 构建时间\\
				\hline		
				6&	2    &方法数 & APK大小 &	构建时间 & 方法数 & APK大小 & 构建时间\\
				\hline			
				7&	2    &方法数 & APK大小 &	构建时间 & 方法数 & APK大小 & 构建时间\\
				\hline		
				8&	2   &方法数 & APK大小 &	构建时间 & 方法数 & APK大小 & 构建时间\\
				\hline	
				9&	2   &方法数 & APK大小 &	构建时间 & 方法数 & APK大小 & 构建时间\\
				\hline	
				10&	2   &方法数 & APK大小 &	构建时间 & 方法数 & APK大小 & 构建时间\\
				\hline
				
				
				
			\end{tabular}
			
			%		\begin{tablenote}
			%		\end{tablenote}
			
		\end{threeparttable}
	}
\end{table*}


}

\eat{

\subsection{日志效率对比实验}

由于RunDroid需要将应用程序运行时的方法执行信息以日志的形式记录下来,而日志记录行为是磁盘读写(I/O)操作,在一定程度上属于耗时操作。
经过技术调研,我们发现Android平台上可行的日志记录方案分为以下几种:
基于Android Log机制的原生日志系统Logcat、基于Java File API的文件读写方案以及基于MMap内存映射机制实现的日志读写方案。


\begin{figure*}[!ht]
	\centering
	
	\resizebox{\textwidth}{!}{
		\begin{threeparttable}[b]
			
			\begin{tabular}{|c|c|c|c|c|c|c|c|}
				\hline
				
				\multirow{2}{0.03\linewidth}{序号}	&\multirow{2}{0.2\linewidth}{\centering{应用名}}& \multicolumn{3}{c}{未采用插桩方案}&  \multicolumn{3}{|c|}{采用插桩方案 }\\
				\cline{3-8}
				
				&	& 方法数 & APK大小 &	构建时间 &方法数 & APK大小 & 构建时间\\
				\hline
				
				1&	2   &方法数 & APK大小 &	构建时间 & 方法数 & APK大小 & 构建时间\\
				\hline
				2&	2   &方法数 & APK大小 &	构建时间 & 方法数 & APK大小 & 构建时间\\
				\hline
				3&	2    &方法数 & APK大小 &	构建时间 & 方法数 & APK大小 & 构建时间\\
				\hline
				4&	2    &方法数 & APK大小 &	构建时间 & 方法数 & APK大小 & 构建时间\\
				\hline
				5&	2    &方法数 & APK大小 &	构建时间 & 方法数 & APK大小 & 构建时间\\
				\hline		
				6&	2    &方法数 & APK大小 &	构建时间 & 方法数 & APK大小 & 构建时间\\
				\hline			
				7&	2    &方法数 & APK大小 &	构建时间 & 方法数 & APK大小 & 构建时间\\
				\hline		
				8&	2   &方法数 & APK大小 &	构建时间 & 方法数 & APK大小 & 构建时间\\
				\hline	
				9&	2   &方法数 & APK大小 &	构建时间 & 方法数 & APK大小 & 构建时间\\
				\hline	
				10&	2   &方法数 & APK大小 &	构建时间 & 方法数 & APK大小 & 构建时间\\
				\hline
				
				
				
			\end{tabular}
			
			%		\begin{tablenote}
			%		\end{tablenote}
			
		\end{threeparttable}
	}


\caption{构建效率对比实验}

\label{tbl:logResult}
\end{figure*}


为了评估以上几种日志的执行效率,本节进行了如下实验:
记录不同记录次数下,记录规格大小不同(\todo{数据的大小在10K $\sim$ 20K之间 })的数据所需要的时间。
本节实验主要在移动设备上进行。
上述实验过程各个日志方案的时间消耗统计结果如 \autoref{tbl:logResult} 所示。
从\autoref{tbl:logResult}中,我们可以看出大部分情况下,Logcat、Java File API 、 MMap三种技术方案的效率依次提升,Logcat效率最低,MMap效率最高。
经过分析,我们发现Android日志系统Logcat虽然通过内存缓冲区加速日志处理,
但是,每一次日志记录操作,Logcat都会在用户空间和内核空间之间切换,减低了日志记录效率。
Java File API方法在执行过程中,不仅涉及到了用户态-内核态之间的切换,内存-页缓存-磁盘三者之间的数据拷贝在一定程度上也增加了整体上的开销。
而MMap方案只需要一次系统调用,避免了过多数据拷贝,执行效率最高。
因此在RunDroid最终实现上,我们基于MMap内存映射机制实现了日志读写方案。


\subsection{运行效率对比实验}


\begin{figure*}[!ht]
	\centering
	
	\resizebox{\textwidth}{!}{
		\begin{threeparttable}[b]
			
			\begin{tabular}{|c|c|c|c|c|c|c|c|}
				\hline
				
				\multirow{2}{0.03\linewidth}{序号}	&\multirow{2}{0.2\linewidth}{\centering{应用名}}& \multicolumn{3}{c}{未采用插桩方案}&  \multicolumn{3}{|c|}{采用插桩方案 }\\
				\cline{3-8}
				
				&	& 方法数 & APK大小 &	构建时间 &方法数 & APK大小 & 构建时间\\
				\hline
				
				1&	2   &方法数 & APK大小 &	构建时间 & 方法数 & APK大小 & 构建时间\\
				\hline
				2&	2   &方法数 & APK大小 &	构建时间 & 方法数 & APK大小 & 构建时间\\
				\hline
				3&	2    &方法数 & APK大小 &	构建时间 & 方法数 & APK大小 & 构建时间\\
				\hline
				4&	2    &方法数 & APK大小 &	构建时间 & 方法数 & APK大小 & 构建时间\\
				\hline
				5&	2    &方法数 & APK大小 &	构建时间 & 方法数 & APK大小 & 构建时间\\
				\hline		
				6&	2    &方法数 & APK大小 &	构建时间 & 方法数 & APK大小 & 构建时间\\
				\hline			
	
				
				
				
			\end{tabular}
			
			%		\begin{tablenote}
			%		\end{tablenote}
			
		\end{threeparttable}
	}
	
	
	\caption{运行效率对比实验}
	
	\label{tbl:hookResult}
\end{figure*}
}



\section{应用结果展示}  
%\label{chp:display}

\begin{figure}
	\centering
	\begin{lstlisting}[language=Java]
package cn.mijack.rundroidtest;

public class MainActivity extends Activity implements View.OnClickListener {
	Button button1,button2,button3;
	Handler handler = new Handler() {
		public void handleMessage(Message msg) {
			if (msg.what == 1) {
				Toast.makeText(MainActivity.this, "handle", Toast.LENGTH_SHORT).show();
			}
		}
	};
	protected void onCreate(Bundle savedInstanceState) {
		super.onCreate(savedInstanceState);
		setContentView(R.layout.activity_main);
		button1 = findViewById(R.id.button1);
		button1.setOnClickListener(this);
		// button2、button3进行相同的操作
	}
	public void onClick(View view) {
		switch (view.getId()) {
			case R.id.button1:
				doHandleButton1();
				return;
				// button2、button3进行相同的操作
		}
	}
	public void doHandleButton1() {
		int fibonacci = doFibonacci(4);
		Toast.makeText(this, "fibonacci: " + fibonacci, Toast.LENGTH_SHORT).show();
	}
	private int doFibonacci(int i) {
		if (i < 0)    return -1;   
		if (i == 1 || i == 0)   return 1; 
		return doFibonacci(i - 1) + doFibonacci(i - 2);
	}
	public void doHandleButton2() {
		Message message = Message.obtain();
		message.what = 1;
		handler.sendMessage(message);
	}
	public void doHandleButton3() {
		Intent intent = new Intent(this, NewActivity.class);
		startActivity(intent);
	}
}\end{lstlisting}
%	\vspace{-9px}
	\caption{Example code for the Android execution model}
	\label{fig:code_demo}
\end{figure}

\subsection{函数调用图的构建结果展示}

斐波拉契数列


\eat{
	
	\begin{figure*}[ht]
		\centering
		\includegraphics[width=0.65\textwidth]{./Figures/code-Fibonacci.png}
		\caption{斐波拉契数列相关的代码实现}
		\label{fig:code-Fibonacci}
\end{figure*}
}
\begin{figure*}[ht]
\centering
\includegraphics[width=\textwidth]{./Figures/FlowDroid-Fibonacci.png}
\caption{斐波拉契数列-FlowDroid生成的调用图}
\label{fig:flowdroid-result-Fibonacci}
\end{figure*}

\subsection{Activity的生命周期效果展示}



\begin{figure*}[ht]
	\centering
	\includegraphics[height=0.45\textheight]{./Figures/flowdroid-dummyMainMethod.png}
	\caption{dummyMainMethod-FlowDroid生成的调用图}
	\label{fig:flowdroid-result-lifecycle}
\end{figure*}


\subsection{多线程触发关系效果展示}

\begin{figure*}[ht]
	\centering
	\includegraphics[width=\textwidth]{./Figures/FlowDroid-handler.png}
	\caption{Handler-FlowDroid生成的调用图}
	\label{fig:flowdroid-result-handler}
\end{figure*}
 \section{本章小结}
 
 本章中,我们将环绕着构建效率、日志效率、运行效率等三个方面对RunDroid角进行系统性能测试和评估。
 并结合实验结果阐述了采用源代码插桩、基于MMap的日志方案等原因。
 
 将展示RunDroid系统生成的函数调用图的运行结果,并对函数调用图进行详细的阐述。
