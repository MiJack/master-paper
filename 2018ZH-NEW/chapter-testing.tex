\chapter{系统测试}  
\label{chp:testing}

本章中,我们将环绕着日志效率、运行效率等方面对RunDroid角进行系统测试和评估。
\section{实验设置}

实验相关配置的环境如下:
PC端对应的型号是ThinkPad E430,对应的CPU型号为Intel 酷睿i5 3210M(双核),内存为12GB。
手机型号为小米 MI 5,对应处理器型号为骁龙820(四核心,最高主频 1.8GHz),RAM为3GB,对应的系统版本为Android 6.0,内置Xposed运行环境。


\eat{


\subsection{构建效率对比实验}
为了验证源代码插桩前后对应用程序的影响,我们从开源社区F-Droid\footnote{https://f-droid.org}中随机选取了\todo{十余}个Android App,比较源代码插桩前后应用构建的耗时时间,
以评估源代码插桩对应用构建的影响。

对于每个应用,我们采用Gradle进行APK构建,通过源代码插桩前后的APK大小变化、方法数变化以及APK构建整体耗时等指标。最终我们得到的结果如 \autoref{tbl:buildResult} 所示。
从\autoref{tbl:buildResult}中,我们发现,源代码插桩前后得到的APK文件方法数变化幅度不大,基本对应的数量变化在10个左右,APK大小变化在\todo{一个百分比范围},构建时间的变化大都在\todo{一个百分比范围}上下。

由于RunDroid采用的源代码插桩方案,通过分析应用程序源代码中各个方法的语法结构,辨识出方法执行过程的方法入口和所有的方法出口,并在这些位置上提交日志代码插桩。
整个过程中,RunDroid引入了有限数量的方法,使得应用的构建耗时无明显的增加。
而文献\cite{van2013dynamic}基于AspectJ的字节码插桩方案,对于每一个方法均会在APK中声明多个新的方法(即AOP中的before通知、after通知等),以保证方法执行前后可以输出相应的信息。
该方法最大的劣势在于对于一些大型的Android应用,该方案引入的大量方法,很容易突破Android系统的方法数限制(即65536),使得应用整体的构建不可进行,APK无法构建成功。
而本方案在这个方面并没有太大的限制。


\begin{table*}[!ht]
	\centering
	
	\caption{构建效率对比实验}
	
	\label{tbl:buildResult}
	
	
	\resizebox{\textwidth}{!}{
		\begin{threeparttable}[b]
			
			\begin{tabular}{|c|c|c|c|c|c|c|c|}
				\hline
				
				\multirow{2}{0.03\linewidth}{序号}	&\multirow{2}{0.2\linewidth}{\centering{应用名}}& \multicolumn{3}{c}{未采用插桩方案}&  \multicolumn{3}{|c|}{采用插桩方案 }\\
				\cline{3-8}
				
				&	& 方法数 & APK大小 &	构建时间 &方法数 & APK大小 & 构建时间\\
				\hline
				
				1&	2   &方法数 & APK大小 &	构建时间 & 方法数 & APK大小 & 构建时间\\
				\hline
				2&	2   &方法数 & APK大小 &	构建时间 & 方法数 & APK大小 & 构建时间\\
				\hline
				3&	2    &方法数 & APK大小 &	构建时间 & 方法数 & APK大小 & 构建时间\\
				\hline
				4&	2    &方法数 & APK大小 &	构建时间 & 方法数 & APK大小 & 构建时间\\
				\hline
				5&	2    &方法数 & APK大小 &	构建时间 & 方法数 & APK大小 & 构建时间\\
				\hline		
				6&	2    &方法数 & APK大小 &	构建时间 & 方法数 & APK大小 & 构建时间\\
				\hline			
				7&	2    &方法数 & APK大小 &	构建时间 & 方法数 & APK大小 & 构建时间\\
				\hline		
				8&	2   &方法数 & APK大小 &	构建时间 & 方法数 & APK大小 & 构建时间\\
				\hline	
				9&	2   &方法数 & APK大小 &	构建时间 & 方法数 & APK大小 & 构建时间\\
				\hline	
				10&	2   &方法数 & APK大小 &	构建时间 & 方法数 & APK大小 & 构建时间\\
				\hline
				
				
				
			\end{tabular}
			
			%		\begin{tablenote}
			%		\end{tablenote}
			
		\end{threeparttable}
	}
\end{table*}


}
\subsection{日志效率对比实验}

由于RunDroid需要将应用程序运行时的方法执行信息以日志的形式记录下来,而日志记录行为是磁盘读写(I/O)操作,在一定程度上属于耗时操作。
经过技术调研,我们发现Android平台上可行的日志记录方案分为以下几种:
基于Android Log机制的原生日志系统Logcat、基于Java File API的文件读写方案以及基于MMap内存映射机制实现的日志读写方案。


\begin{figure*}[!ht]
	\centering
	
	\resizebox{\textwidth}{!}{
		\begin{threeparttable}[b]
			
			\begin{tabular}{|c|c|c|c|c|c|c|c|}
				\hline
				
				\multirow{2}{0.03\linewidth}{序号}	&\multirow{2}{0.2\linewidth}{\centering{应用名}}& \multicolumn{3}{c}{未采用插桩方案}&  \multicolumn{3}{|c|}{采用插桩方案 }\\
				\cline{3-8}
				
				&	& 方法数 & APK大小 &	构建时间 &方法数 & APK大小 & 构建时间\\
				\hline
				
				1&	2   &方法数 & APK大小 &	构建时间 & 方法数 & APK大小 & 构建时间\\
				\hline
				2&	2   &方法数 & APK大小 &	构建时间 & 方法数 & APK大小 & 构建时间\\
				\hline
				3&	2    &方法数 & APK大小 &	构建时间 & 方法数 & APK大小 & 构建时间\\
				\hline
				4&	2    &方法数 & APK大小 &	构建时间 & 方法数 & APK大小 & 构建时间\\
				\hline
				5&	2    &方法数 & APK大小 &	构建时间 & 方法数 & APK大小 & 构建时间\\
				\hline		
				6&	2    &方法数 & APK大小 &	构建时间 & 方法数 & APK大小 & 构建时间\\
				\hline			
				7&	2    &方法数 & APK大小 &	构建时间 & 方法数 & APK大小 & 构建时间\\
				\hline		
				8&	2   &方法数 & APK大小 &	构建时间 & 方法数 & APK大小 & 构建时间\\
				\hline	
				9&	2   &方法数 & APK大小 &	构建时间 & 方法数 & APK大小 & 构建时间\\
				\hline	
				10&	2   &方法数 & APK大小 &	构建时间 & 方法数 & APK大小 & 构建时间\\
				\hline
				
				
				
			\end{tabular}
			
			%		\begin{tablenote}
			%		\end{tablenote}
			
		\end{threeparttable}
	}


\caption{构建效率对比实验}

\label{tbl:logResult}
\end{figure*}


为了评估以上几种日志的执行效率,本节进行了如下实验:
记录不同记录次数下,记录规格大小不同(\todo{数据的大小在10K $\sim$ 20K之间 })的数据所需要的时间。
本节实验主要在移动设备上进行。
上述实验过程各个日志方案的时间消耗统计结果如 \autoref{tbl:logResult} 所示。
从\autoref{tbl:logResult}中,我们可以看出大部分情况下,Logcat、Java File API 、 MMap三种技术方案的效率依次提升,Logcat效率最低,MMap效率最高。
经过分析,我们发现Android日志系统Logcat虽然通过内存缓冲区加速日志处理,
但是,每一次日志记录操作,Logcat都会在用户空间和内核空间之间切换,减低了日志记录效率。
Java File API方法在执行过程中,不仅涉及到了用户态-内核态之间的切换,内存-页缓存-磁盘三者之间的数据拷贝在一定程度上也增加了整体上的开销。
而MMap方案只需要一次系统调用,避免了过多数据拷贝,执行效率最高。
因此在RunDroid最终实现上,我们基于MMap内存映射机制实现了日志读写方案。


\subsection{运行效率对比实验}


\begin{figure*}[!ht]
	\centering
	
	\resizebox{\textwidth}{!}{
		\begin{threeparttable}[b]
			
			\begin{tabular}{|c|c|c|c|c|c|c|c|}
				\hline
				
				\multirow{2}{0.03\linewidth}{序号}	&\multirow{2}{0.2\linewidth}{\centering{应用名}}& \multicolumn{3}{c}{未采用插桩方案}&  \multicolumn{3}{|c|}{采用插桩方案 }\\
				\cline{3-8}
				
				&	& 方法数 & APK大小 &	构建时间 &方法数 & APK大小 & 构建时间\\
				\hline
				
				1&	2   &方法数 & APK大小 &	构建时间 & 方法数 & APK大小 & 构建时间\\
				\hline
				2&	2   &方法数 & APK大小 &	构建时间 & 方法数 & APK大小 & 构建时间\\
				\hline
				3&	2    &方法数 & APK大小 &	构建时间 & 方法数 & APK大小 & 构建时间\\
				\hline
				4&	2    &方法数 & APK大小 &	构建时间 & 方法数 & APK大小 & 构建时间\\
				\hline
				5&	2    &方法数 & APK大小 &	构建时间 & 方法数 & APK大小 & 构建时间\\
				\hline		
				6&	2    &方法数 & APK大小 &	构建时间 & 方法数 & APK大小 & 构建时间\\
				\hline			
				7&	2    &方法数 & APK大小 &	构建时间 & 方法数 & APK大小 & 构建时间\\
				\hline		
				8&	2   &方法数 & APK大小 &	构建时间 & 方法数 & APK大小 & 构建时间\\
				\hline	
				9&	2   &方法数 & APK大小 &	构建时间 & 方法数 & APK大小 & 构建时间\\
				\hline	
				10&	2   &方法数 & APK大小 &	构建时间 & 方法数 & APK大小 & 构建时间\\
				\hline
				
				
				
			\end{tabular}
			
			%		\begin{tablenote}
			%		\end{tablenote}
			
		\end{threeparttable}
	}
	
	
	\caption{运行效率对比实验}
	
	\label{tbl:hookResult}
\end{figure*}

\chapter{应用结果展示}  
\label{chp:display}


\section{函数调用图的构建结果展示}
\section{Activity的生命周期效果展示}
\section{多线程触发关系效果展示}
% \section{应用程序的状态效果展示}

 \section{本章小结}
