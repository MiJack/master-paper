
%============================= 宏包 ================================%
\usepackage{dsfont}
\usepackage[CJKbookmarks]{hyperref}
\usepackage{shortvrb,indentfirst,ulem,makeidx}
\usepackage{multirow}
\usepackage{setspace}%设置间距
\usepackage{fancyhdr}
\usepackage{graphics}
\usepackage{indentfirst,latexsym,graphicx,colortbl}
\usepackage[boxruled,algoruled,algochapter]{algorithm2e}
\usepackage{bm}                     % 处理数学公式中的黑斜体的宏包
\usepackage{amsmath}                % AMSLaTeX宏包 用来排出更加漂亮的公式
%\usepackage{amsfonts}
\usepackage{amssymb}                % AMSLaTeX宏包 用来排出更加漂亮的公式
\usepackage{stmaryrd}
\usepackage{mathrsfs}               % 不同于\mathcal or \mathfrak 之类的英文花体字体
\usepackage[subnum]{cases}
%\usepackage{zed-csp}
\usepackage{subfig}
%\usepackage{times}
\usepackage{amsthm}
\pagestyle{fancy}
\usepackage{fancyvrb}
\usepackage{listings}
\usepackage{caption}
\usepackage{algorithmic} 
%\usepackage{subcaption}
\usepackage{etoolbox}% http://ctan.org/pkg/etoolbox
\usepackage{leading}
\usepackage{titlesec}
%\usepackage{CJKnumb}
%\usepackage{titletoc}
\usepackage[T1]{fontenc}
\usepackage{tikz}
\usetikzlibrary{calc}
\usetikzlibrary{arrows}
\usetikzlibrary{positioning}
\usetikzlibrary{automata}
\newcommand{\mytilde}{$\sim$}
\fancyhead{} % clear all fields
\fancyhead[RO,LE]{\bfseries 华东师范大学博士学位论文}
%\fancyfoot[LE,RO]{\thepage}
%\fancyhead[RE]{\small \CAST@value@titlemark}
\fancyhead[LO]{\small \leftmark} \fancyhead[RE]{\small \leftmark}
%\renewcommand{\headrulewidth}{0.4pt}
\fancyfoot[CO,CE]{\thepage}


\newcommand\Tstrut{\rule{0pt}{2.6ex}}  

%==============================定义文字块 ================================%
\newcommand{\changefont}[3]{
\fontfamily{#1} \fontseries{#2} \fontshape{#3} \selectfont}

%举例%%%%%%%%%%%%%%%%%%%%%%%%%
\newtheorem{example_type}{Theorem}[section]
\newtheorem{example_def}[example_type]{例\hspace*{0.5mm}}
\newcommand{\example}[1]{\begin{example_def}#1\end{example_def}}

%定义%%%%%%%%%%%%%%%%%%%%%%%%%
\newtheorem{define_type}{Theorem}[section]
\newtheorem{define_def}[define_type]{定义\hspace*{0.5mm}}
\newcommand{\define}[1]{\begin{define_def}#1\end{define_def}}

%定理%%%%%%%%%%%%%%%%%%%%%%%%%
\newtheorem{theorem_type}{Theorem}[section]
\newtheorem{theorem_def}[theorem_type]{定理\hspace*{0.5mm}}
\newcommand{\theorem}[1]{\begin{theorem_def}#1\end{theorem_def}}

%命题%%%%%%%%%%%%%%%%%%%%%%%%%
\newtheorem{proposition_type}{Theorem}[section]
\newtheorem{proposition_def}[proposition_type]{命题\hspace*{0.5mm}}
\newcommand{\proposition}[1]{\begin{proposition_def}#1\end{proposition_def}}

\renewcommand{\algorithmcfname}{算法}
\renewcommand{\algorithmicrequire}{ \textbf{输入:}} %Use Input in the format of Algorithm  
\renewcommand{\algorithmicensure}{ \textbf{输出:}} %UseOutput in the format of Algorithm  
\tikzstyle{auto}=[draw,text width=7em,text centered, minimum height=2.5em]
\tikzstyle{sc} = [auto, minimum height=3em, rounded corners]
\tikzstyle{autoo}=[draw,text width=7.7em,text centered, minimum height=2.5em]
\tikzstyle{scc} = [autoo, minimum height=3em, rounded corners]
\tikzstyle{autos}=[draw,text width=3em,text centered, minimum height=2.5em]
\tikzstyle{scs} = [autos, minimum height=3em, rounded corners]

\tikzstyle{mido}=[draw,text width=6.5em,text centered, minimum height=2.5em]
\tikzstyle{middlen} = [mido, minimum height=3em, rounded corners]

\tikzstyle{smido}=[draw,text width=5.3em,text centered, minimum height=2.5em]
\tikzstyle{smiddlen} = [smido, minimum height=3em, rounded corners]

\tikzstyle{bigautoo}=[draw,text width=11em,text centered, minimum height=2.5em]
\tikzstyle{bigscc} = [bigautoo, minimum height=3em, rounded corners]




\definecolor{mygray}{rgb}{0.5,0.5,0.5}
\lstset{basicstyle=\footnotesize,numbersep=-6pt,numbers=left,
escapeinside={||},
mathescape=true,frame=tlrb,xleftmargin=\fboxsep,xrightmargin=\fboxsep}

\lstdefinestyle{ApricotStyle}{basicstyle=\scriptsize,numbersep=-6pt,numbers=left, numberstyle=\scriptsize\color{mygray},
escapeinside={||},mathescape=true,frame=tlrb,xleftmargin=\fboxsep,xrightmargin=\fboxsep}
\lstdefinestyle{Apricot} {style=ApricotStyle}


\lstdefinestyle{STTTStyle}{stringstyle=\ttfamily,numbersep=-5pt,numbers=left, numberstyle=\scriptsize\color{mygray},
escapeinside={||},mathescape=true,frame=tlrb,xleftmargin=\fboxsep,xrightmargin=\fboxsep}
\lstdefinestyle{STTT} {style=STTTStyle}

%%%%%%%%%%%%%%%%%%%%%%%%%%%%

%==============================自定义文字块 ================================%
\newcommand{\eqdef}{=_{df}}
\newcommand{\until}{{\textcolor{blue}{\bf \tt  until~}}}
\newcommand{\init}{{\textcolor{blue}{~\bf \tt  init~}}}
\newcommand{\while}{{\textcolor{blue}{\bf \tt  while~}}}
\newcommand{\when}{{\textcolor{blue}{\bf \tt  when}}}
\newcommand{\wdo}{{\textcolor{blue}{~\bf \tt  do~}}}
\newcommand{\wod}{{\textcolor{blue}{\bf \tt  od}}}
\newcommand{\skips}{{\textcolor{blue}{\bf \tt  skip}}}
\newcommand{\suspend}{{\textcolor{blue}{\bf \tt  suspend}}}
\newcommand{\out}{{\textcolor{blue}{\bf \tt  out}}}
\newcommand{\chaos}{{\textcolor{blue}{\bf \tt  chaos}}}
\newcommand{\stops}{{\textcolor{blue}{\bf \tt  stop}}}
\newcommand{\glb}{{\textcolor{black}{\bf \tt  glb}}}
\newcommand{\maximum}{{\textcolor{black}{\bf \tt  max}}}
\newcommand{\minimum}{{\textcolor{black}{\bf \tt  min}}}
\newcommand{\term}{{\textcolor{blue}{\bf \tt  term}}}
\newcommand{\stable}{{\textcolor{blue}{\bf \tt  stable}}}
\newcommand{\diver}{{\textcolor{blue}{\bf \tt  div}}}
\newcommand{\post}[2]{\langle {#1}\bowtie{#2}  \rangle}
\newcommand{\true}{{\textcolor{black}{\bf \tt  T}}}
\newcommand{\false}{{\textcolor{black}{\bf \tt  F}}}
\newcommand{\sep}{{\bf  ~\mid~}}
\newcommand{\wait}{{\textcolor{blue}{\bf \tt  wait}}}
\newcommand{\choice}[3]{#1 ~\langle~ #2 ~\rangle~ #3}
\newcommand{\then}{{\textcolor{blue}{\bf \tt  then}}}
\newcommand{\float}{{\textcolor{blue}{\bf \tt  float}}}
\newcommand{\intType}{{\textcolor{blue}{\bf \tt  int~}}}
\newcommand{\Boolean}{{\textcolor{blue}{\bf \tt  boolean~}}}
\newcommand{\Signal}{{\textcolor{blue}{\bf \tt  Signal~}}}
\newcommand{\inin}{{\textcolor{blue}{\bf \tt  in}}}
\newcommand{\Template}{{\textcolor{blue}{\bf \tt  Template~}}}
\newcommand{\Main}{{\textcolor{blue}{\bf \tt  Main~}}}
\newcommand{\final}{{\textcolor{blue}{\bf \tt  final~}}}
\newcommand{\parr}{{\textcolor{blue}{\bf \tt  \mid\mid}}}
\newcommand{\derivative}{{\textcolor{blue}{\bf \tt  dot~}}}
\newcommand{\lrf}{$\mathcal{L}_{\mathbb{R}_{\mathcal{F}}}$}
\newcommand{\hoare}[3]{\{#1\} ~#2~ \{#3\}}
%示例
%\newcommand{\imp}{\Rightarrow}
%\newtheorem{thm1}{Theorem}[part]
%\newtheorem{thm2}{Theorem}[section]
%\newtheorem{thm3}{Theorem}[subsection]

\newcommand{\hrl}[1]{
\abovedisplayskip=-10pt
\belowdisplayskip=10pt
\begin{align*}
#1
\end{align*}
}


%%%%%%%%%%%%%%%%正文章节字体%%%%%%%%%%%%%%%%%%%%%
\newcommand{\xiaosan}{\fontsize{15pt}{22pt}\selectfont}    % 小三, 1.5倍行距
\newcommand{\xiaoer}{\fontsize{18pt}{27pt}\selectfont}    % 小二, 1.5倍行距
\renewcommand{\LARGE}{\fontsize{16pt}{\baselineskip}\selectfont}
\newcommand{\sihao}{\fontsize{14pt}{\baselineskip}\selectfont}
\newcommand{\xiaosihao}{\fontsize{12pt}{\baselineskip}\selectfont}
\titleformat{\chapter}[hang]{\centering\heiti\LARGE\bfseries}{第\chaptername章}{1em}{}{}
\titleformat{\section}[hang]{\centering\heiti\sihao\bfseries}{\thesection}{1em}{}{}
\titleformat{\subsection}[hang]{\xiaosihao\heiti\bfseries}{\thesubsection}{1em}{}{}

%%%%%%%%%%%%%%%%%%%%%%%%%%%%%%%%%%%%%


\makeatletter
\renewcommand\normalsize{%
   \@setfontsize\normalsize{12.2pt}{14pt}
   \abovedisplayskip 3\p@ \@plus5\p@ \@minus5\p@
   \abovedisplayshortskip \z@ \@plus3\p@
   \belowdisplayshortskip 6\p@ \@plus3\p@ \@minus3\p@
   \belowdisplayskip 3\p@ \@plus5\p@ \@minus5\p@
   \let\@listi\@listI}
\makeatother


%%%%%%%%%%%%%%%%%%%设置脚注行距%%%%%%%%%%%%%%
\makeatletter
\patchcmd{\@makefntext}% <cmd>
  {#1}% <search>
  {\leading{15pt}#1}% <replace>
  {}{}% <success><failure>
\makeatother

%%%%%%%%%%%%设置caption行距%%%%%%%%%%%%%%
\makeatletter
\patchcmd{\@makecaption}% <cmd>
  {#2}% <search>
  {\leading{17pt}#2}% <replace>
  {}{}% <success><failure>
\makeatother
%%% ----------------------------------------------------------------------

%============================= 版芯控制 ================================%
\voffset-10mm \textwidth=150mm \textheight=230mm \headwidth=150mm
\rightmargin=10mm
%                                                                       %
\setlength{\oddsidemargin}{0.45in}
\setlength{\evensidemargin}{-0.05in}

%============================= 页面设置 ================================%
%-------------------- 定义页眉和页脚 使用fancyhdr 宏包 -----------------%
% 定义页眉与正文间双隔线
\newcommand{\makeheadrule}{%
\makebox[0pt][l]{\rule[.7\baselineskip]{\headwidth}{0.4pt}}%
\rule[0.85\baselineskip]{\headwidth}{0.4pt} \vskip-.8\baselineskip}
\makeatletter
\renewcommand{\headrule}{%
{\if@fancyplain\let\headrulewidth\plainheadrulewidth\fi
\makeheadrule}} \makeatother
\newcommand{\adots}{\mathinner{\mkern 2mu%
\raisebox{0.1em}{.}\mkern 2mu\raisebox{0.4em}{.}%
\mkern2mu\raisebox{0.7em}{.}\mkern 1mu}}
\makeindex